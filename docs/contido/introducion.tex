\chapter{Introdución}
\label{chap:introducion}

\lettrine{E}{ste} traballo forma parte da materia de Procesamento dixital da información como un TFG breve. O traballo consiste na creación dun proxecto sinxelo que estea relacionado cos contidos da materia, como por exemplo adquisición e conversión de sinais, procesamento de sinais dixitais e filtrado de sinais, e tamén co resto de contidos da mención. Ademais da relación ca materia, un dos principais puntos do traballo era que o proxecto puidese ser utilizado por nenos e persoas con discapacidades cognitivas. Debido a isto podemos falar deste proxecto como aprendizaxe-servizo.
\newline\newline
O aprendizaxe-servizo é unha metodoloxía empregada en pedagoxía que mestura e xunta tanto o aprendizaxe co servizo á comunidade \cite{AprendizaxeServizo}. Este método fomenta a calidade e a ética da formación co compromiso social e a construción de sociedades máis solidarias.
\newline\newline
O proxecto que se quere crear que mesture estes dous propósitos é un teclado de piano. Este teclado consiste de dúas partes conectadas entre si. A primeira sería un teclado físico creado con un microcontrolador conectado a varios botóns mediante unha protoboard. Este microcontrolador usará un programa que detecte as pulsacións nos botóns e enviará un mensaxe a través do porto serie. A segunda parte é a aplicación de escritorio que se encargará de de recibir estes mensaxes a través do porto serie e reproducir os sons dos animais. Ademais, presentará unha vista onde se poderá elixir entre os diferentes animais dispoñibles e facer sonar o teclado usando ás teclas do teclado do ordenador. Deste xeito, conseguimos unha maior accesibilidade xa que o proxecto poderase usar tamén sen o hardware específico, soamente cun ordenador persoal corrente.
\newline\newline

%TFG BREVE: 
%
%La introducción debe explicar claramente cuál es el interés del trabajo realizado.
% En concreto, dada la temática abordada en este proyecto, debes explicar la importancia de la tecnología para la integración de personas con discapacidad. Es importante incluir citas bibliograficas utilizando este formato: \cite{TauriAsyncRustProcess}. 


 \section{Obxectivos}
Para acadar 

\section{Organización da memoria}
Y una sección donde se explique brevemente el contenido de los otros capítulos de la memoria. 

El resto de la memoria está organizado como sigue:
\begin{itemize}
\item El capítulo  \ref{cap:desarrollo} recoge...
\item El capítulo \ref{cap:conclusiones} contiene las conclusiones del proyecto y platea futuras líneas de mejorar.
\end{itemize}
