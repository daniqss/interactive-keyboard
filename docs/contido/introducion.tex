\chapter{Introdución}
\label{chap:introducion}

\lettrine{E}{ste} traballo forma parte da materia de Procesamento dixital da información como un TFG breve. O traballo consiste na creación dun proxecto sinxelo que estea relacionado cos contidos da materia, como por exemplo adquisición e conversión de sinais, procesamento de sinais dixitais e filtrado de sinais, e tamén co resto de contidos da mención. Ademais da relación ca materia, un dos principais puntos do traballo era que o proxecto puidese ser utilizado por nenos e persoas con discapacidades cognitivas. Debido a isto podemos falar deste proxecto como aprendizaxe-servizo.
\newline\newline
O aprendizaxe-servizo é unha metodoloxía empregada en pedagoxía que mestura e xunta tanto o aprendizaxe co servizo á comunidade \cite{AprendizaxeServizo}. Este método fomenta a calidade e a ética da formación co compromiso social e a construción de sociedades máis solidarias.
\newline\newline
O proxecto que se quere crear que mesture estes dous propósitos é un teclado de piano. Este teclado consiste de dúas partes conectadas entre si. A primeira sería un teclado físico creado con un microcontrolador conectado a varios botóns mediante unha protoboard. Este microcontrolador usará un programa que detecte as pulsacións nos botóns e enviará un mensaxe a través do porto serie. A segunda parte é a aplicación de escritorio que se encargará de de recibir estes mensaxes a través do porto serie e reproducir os sons dos animais. Ademais, presentará unha vista onde se poderá elixir entre os diferentes animais dispoñibles e facer sonar o teclado usando ás teclas do teclado do ordenador. Deste xeito, conseguimos unha maior accesibilidade xa que o proxecto poderase usar tamén sen o hardware específico, soamente cun ordenador persoal corrente.
\newline\newline

 \section{Obxectivos}
Para acadar o resultado desexado ideáronse diferentes obxectivos.
\begin{itemize}
\item Crear un sistema de procesado de son que nos permita crear as diferentes notas do piano baseando o son no dunha serie de animais. 
\item Crear un piano que utilice un sistema hardware creado a partir dun microcontrolador que sirva para recibir as pulsacións do usuario e comunicalas a través do porto serie.
\item Crear unha aplicación que cumpra tres subobxectivos
    \begin{itemize}
        \item Mostre unha vista que permita ao usuario interactuar ca aplicación dunha forma accesible e intuitiva.
        \item Integre o sistema de procesado de son para que os outros sistemas poidan acceder a el e utilizalo. 
        \item Integrar o sistema hardware para este poida usar o sistema de procesado de son, recibindo as comunicacións do sistema hardware a través do porto serie.
    \end{itemize} 
\end{itemize}
    


\section{Organización da memoria}
A continuación vou a explicar como se estrutura o contida desta memoria comentando cal é o cometido dos seguintes capítulos.
\begin{itemize}
\item O capítulo  \ref{cap:desarrollo} recolle os pasos seguidas durante a realización do traballo, dende a elección das tecnoloxías empregadas, ata a resolución dos diferentes problemas a resolver que xurdiron durante a implementación.
\item O capítulo \ref{cap:conclusiones} contén as conclusións ás que se chegaron ao rematar o proxecto e expón futuras liñas de mellorar.
\end{itemize}
