\chapter{Desarrollo}
\label{cap:desarrollo}

TFG BREVE: Es conveniente comenzar el capítulo un un par de párrafos que explique su contenido. Por ejemplo: "En este capítulo se presenta en desarrollo del proyecto realizado. Comenzaremos explicando las tecnologías utilizadas para, posteriormente, detallar ...."

\section{Tecnologías}
En el TFG se suele poner un capítulo completo dedicado a las tecnologías empleadas, pero en el TFG BREVE se hará en esta sección. Se pretende resumir las tecnologías y explicar por qué se han elegido. Hay que evitar hacer copy-paste de web. Además, es importante poner referencias bibliográficas utilizando este formato: \cite{ErlangBook,ErlangWebBook}. 

\section{Desarrollo}
En el TFG suele haber varios capítulos para explicar el desarrollo realizados. Aquí lo haremos en un solo capítulo.

En el caso de proyectos con microcontroladores, se debe indicar claramente el montaje. 

En caso de incluir figuras, deben referenciarse de esta forma "la figura~\ref{fig:exemplo} ". Todas las figuras deben estar referenciadas en el texto.
Se recomienda poner todas las figuras en el directorio \texttt{imaxes}.

\begin{figure}[hp!]
  \centering
  \includegraphics[width=0.75\textwidth]{imaxes/udc.png}
  \caption{Título de la figura}
  \label{fig:exemplo}
\end{figure}

\begin{figure}[hp!]
  \centering
  \begin{subfigure}[c]{0.3\textwidth}
    \includegraphics[angle=45,width=\textwidth]{imaxes/udc.png}
    \caption{Título figura 1}
    \label{fig:subfigura-rotada}
  \end{subfigure}
  \hspace{0.1\textwidth}
  \begin{subfigure}[c]{0.3\textwidth}
    \includegraphics[width=\textwidth,height=3cm]{imaxes/udc.png}
    \caption{Título figura 2}
    \label{fig:subfigura-deformada}
  \end{subfigure}
  \caption{Título general de la figura}
  \label{fig:exemplo-subfiguras}
\end{figure}

Se puede añadir partes del código que sean importantes para entender el desarrollo.

Se pueden incluir cuadros (tablas). Los cuadros tienen que estar referenciados en el texto (por ejemplo, el cuadro \ref{tab:exemplo}).

\begin{table}[hp!]
  \centering
  \rowcolors{2}{white}{udcgray!25}
  \begin{tabular}{c|c}
  \rowcolor{udcpink!25}
  \textbf{Título de columna} & \textbf{Outro título de columna} \\\hline
  \textit{Título de fila} & Contido de celda \\
  \textit{Título de fila} & Contido de celda \\
  \textit{Título de fila} & Contido de celda \\
  \textit{Título de fila} & Contido de celda \\
  \textit{Título de fila} & Contido de celda \\
  \textit{Título de fila} & Contido de celda \\
  \end{tabular}
  \caption{Título del cuadro}
  \label{tab:exemplo}
\end{table}



